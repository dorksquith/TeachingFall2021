% preamble and style file for M&R lecture slides
\documentclass[11.5pt,sans,english]{beamer}

\usetheme{EastLansing}
\usecolortheme{lily}

\usepackage[most]{tcolorbox}

\usepackage{verbatim}
%\usepackage{ulem}
%\usepackage{fontawesome}
%\usepackage{tikz}
%\usepackage{pifont}
%\usepackage{tabularx}
\usepackage{array,booktabs,xcolor,colortbl,multirow,rotating,amssymb}
%\usepackage{amsmath}
% \usepackage{vwcol}
% \usepackage[T1]{fontenc}

  
\newcommand\vect[1]{\underline{\mathbf{#1}}}
\newcommand\unitvect[1]{\hat{\boldsymbol{#1}}}
%\newcommand\hatdot[1] { \hat{ \dot{ \boldsymbol{#1} } } }

\newtcbox
{\keyc}{on line,arc=2pt, colback=yellow!30!white, colframe=yellow!30!black, before upper={\rule[-3pt]{0pt}{10pt} },boxrule=1pt,boxsep=0pt,left=6pt,right=6pt,top=2pt,bottom=2pt,}

\newtcbox
{\keyb}{on line,arc=1pt, colback=blue!30!white, colframe=blue!30!black, before upper={\rule[-3pt]{0pt}{10pt} },boxrule=1pt,boxsep=0pt,left=6pt,right=6pt,top=2pt,bottom=2pt,}

\newtcbox
{\keyl}{on line,arc=1pt, colback=pink!30!white, colframe=blue!30!black, before upper={\rule[-3pt]{0pt}{10pt} },boxrule=1pt,boxsep=0pt,left=6pt,right=6pt,top=2pt,bottom=2pt,}

\newtcbox
{\keyw}{on line,arc=1pt, colback=red!30!white, colframe=blue!30!black, before upper={\rule[-3pt]{0pt}{10pt} },boxrule=1pt,boxsep=0pt,left=6pt,right=6pt,top=2pt,bottom=2pt,}

\newtcbox
{\keya}{on line,arc=1pt, colback=purple!30!white, colframe=blue!30!black, before upper={\rule[-3pt]{0pt}{10pt} },boxrule=1pt,boxsep=0pt,left=6pt,right=6pt,top=2pt,bottom=2pt,}

\newtcbox[auto counter,number within=section]
{keyf}
{
enhanced,
on line,
  boxsep=0pt,
  left=6pt,right=6pt,top=2pt,bottom=2pt,
  arc=5pt,
  boxrule=1pt,
  rightrule=38pt,
colback=green!10!white, 
colframe=green!50!black, 
title=\thetcbcounter,
detach title,
overlay unbroken and first ={
    \node[%rotate=90,
          %minimum width=1cm,
          anchor=south,
          font=\sffamily\bfseries\tiny,
          %yshift=-10pt,
          yshift=-5pt,
          xshift=-20pt,
          white]
    at (frame.east) {\thetcbcounter};
  }
}


\usepackage{xcolor}

%\usepackage{hyperref}
%\hypersetup{
%  pdfauthor={Lily Asquith},
%  urlcolor=blue,
%  colorlinks=true,
%  linkcolor=blue,
%  bookmarks=true
%}

%---------------------------------------------%
%              LILY'S COLOURS           %
%---------------------------------------------%
\definecolor{Wash}{RGB}{204,204,204}
%\definecolor{Pinky}{RGB}{254,200,254}%violet
\definecolor{Pinky}{RGB}{219,	240,	253}%violet
\definecolor{Bluey}{RGB}{0,190,255}%deep sky blue
\definecolor{DarkGrey}{RGB}{28,66,137}%dar grey
\definecolor{SussexWhite}{RGB}{253,255,254}%dar grey
\definecolor{LightGray}{RGB}{184,184,255}
\definecolor{YesGreen}{RGB}{0,128,0}
\definecolor{NoRed}{RGB}{250,0,0}



\definecolor{myred}{RGB}{255,153,153}
\definecolor{myorange}{RGB}{255,204,153}
\definecolor{myyellow}{RGB}{255,255,153}
\definecolor{mygreen}{RGB}{153,255,153}
\definecolor{mycyan}{RGB}{153,255,255}
\definecolor{myblue}{RGB}{153,204,255}
\definecolor{myviolet}{RGB}{153,153,255}
\definecolor{mypurple}{RGB}{204,153,255}
\definecolor{mypink}{RGB}{255,204,255}
\definecolor{mycoral}{RGB}{255,153,204}

%-----------------------------------------------------%
%              LILY'S COLUMN TYPES          %
%-----------------------------------------------------%
\newcolumntype{a}{>{\raggedright\arraybackslash}l}	
\newcolumntype{q}{>{\raggedright\arraybackslash}m{8cm}} 

%--------------------------------------------%
%              LILY'S SYMBOLS          %
%--------------------------------------------%
\newcommand{\dfinger}{\large{\textcolor{black}{\ding{43}}}\scriptsize}
\newcommand{\dstar}{\large{\textcolor{black}{\ding{76}}}\scriptsize}
\newcommand{\dwrite}{\large{\textcolor{black}{\ding{45}}}\scriptsize}
\newcommand{\ddiamond}{\small{\textcolor{DarkGrey}{\ding{117}}}\scriptsize}
\newcommand{\ddiamondwhite}{\small{\textcolor{SussexWhite}{\ding{117}}}\scriptsize}
\newcommand{\experiment}{\small{\textcolor{magenta}{\faCogs }}\scriptsize}
\newcommand{\watchit}{\textcolor{blue}{ \faYoutube}}


\makeatletter
\newcommand\notsotiny{\@setfontsize\notsotiny{6.5}{7.5}}
\makeatother


% 
\title[ Mechanics \& Relativity]{Mechanics \& Relativity}
%\subtitle{\textbf{Topic 1: Kinematics }}
\author[Dr Lily Asquith (Lily)]{ Dr Lily Asquith (Lily)}
\date[Week 6: Problems]{Week 6: Problems}
\logo{
\includegraphics[width=1.5cm]{../../utils/uslogo.jpg}
}


\begin{document}


\begin{frame}
\titlepage
\end{frame} 







 \subsection{Problems}
 
 \begin{frame}{Most Hated AP 6.1}
\small

\includegraphics[scale=0.35]{mosthated}

\vspace{10cm}
\end{frame}


\begin{frame}{Most Hated AP 6.2}
\small

\includegraphics[scale=0.35]{mosthated2}

\vspace{10cm}
\end{frame}

\begin{frame}{Most Hated AP 6.3}
\small

\includegraphics[scale=0.35]{mosthated3}

\vspace{10cm}
\end{frame}

\begin{frame}{Problems}
\small
An object, with mass m and speed v relative to an observer, explodes into two pieces, one three times as massive as the other; the explosion takes place in deep space. The less massive piece stops relative to the observer. How much kinetic energy is added to the system during the explosion, as measured in the observer's reference frame?
\vspace{10cm}
\end{frame}


\begin{frame}{Problems}
\small
A uniform soda can of mass 0.10 kg is 10.0 cm tall and filled with 200 g of soda. Then small holes are drilled in the top and bottom (with negligible loss of metal) to drain the soda. What is the height h of the com of the can and contents (a) initially and (b) after the can loses all the soda? 
\vspace{10cm}
\end{frame}

\begin{frame}{Problems}
\small
A 700 g block is released from rest at height $h_0$ above a vertical spring with spring constant k = 400 N/m and negligible mass. The block sticks to the spring and momentarily stops after compressing the spring 19.0 cm. How much work is done (a) by the block on the spring and (b) by the spring on the block? (c) What is the value of $h_0$? (d) If the block were released from height 2$h_0$ above the spring, what would be the maximum compression of the spring?
\vspace{10cm}
\end{frame}

\begin{frame}{Problems}
\small
A completely inelastic collision occurs between two balls of wet putty that move directly toward each other along a vertical axis. Just before the collision, one ball, of mass 3.0 kg, is moving upward at 20 m/s and the other ball, of mass 2.0 kg, is moving downward at 12 m/s. How high do the combined two balls of putty rise above the collision point? (Neglect air drag.)
\vspace{10cm}
\end{frame}


\end{document}