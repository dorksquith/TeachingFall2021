% preamble and style file for M&R lecture slides
\documentclass[11.5pt,sans,english]{beamer}

\usetheme{EastLansing}
\usecolortheme{lily}

\usepackage[most]{tcolorbox}

\usepackage{verbatim}
%\usepackage{ulem}
%\usepackage{fontawesome}
%\usepackage{tikz}
%\usepackage{pifont}
%\usepackage{tabularx}
\usepackage{array,booktabs,xcolor,colortbl,multirow,rotating,amssymb}
%\usepackage{amsmath}
% \usepackage{vwcol}
% \usepackage[T1]{fontenc}

  
\newcommand\vect[1]{\underline{\mathbf{#1}}}
\newcommand\unitvect[1]{\hat{\boldsymbol{#1}}}
%\newcommand\hatdot[1] { \hat{ \dot{ \boldsymbol{#1} } } }

\newtcbox
{\keyc}{on line,arc=2pt, colback=yellow!30!white, colframe=yellow!30!black, before upper={\rule[-3pt]{0pt}{10pt} },boxrule=1pt,boxsep=0pt,left=6pt,right=6pt,top=2pt,bottom=2pt,}

\newtcbox
{\keyb}{on line,arc=1pt, colback=blue!30!white, colframe=blue!30!black, before upper={\rule[-3pt]{0pt}{10pt} },boxrule=1pt,boxsep=0pt,left=6pt,right=6pt,top=2pt,bottom=2pt,}

\newtcbox
{\keyl}{on line,arc=1pt, colback=pink!30!white, colframe=blue!30!black, before upper={\rule[-3pt]{0pt}{10pt} },boxrule=1pt,boxsep=0pt,left=6pt,right=6pt,top=2pt,bottom=2pt,}

\newtcbox
{\keyw}{on line,arc=1pt, colback=red!30!white, colframe=blue!30!black, before upper={\rule[-3pt]{0pt}{10pt} },boxrule=1pt,boxsep=0pt,left=6pt,right=6pt,top=2pt,bottom=2pt,}

\newtcbox
{\keya}{on line,arc=1pt, colback=purple!30!white, colframe=blue!30!black, before upper={\rule[-3pt]{0pt}{10pt} },boxrule=1pt,boxsep=0pt,left=6pt,right=6pt,top=2pt,bottom=2pt,}

\newtcbox[auto counter,number within=section]
{keyf}
{
enhanced,
on line,
  boxsep=0pt,
  left=6pt,right=6pt,top=2pt,bottom=2pt,
  arc=5pt,
  boxrule=1pt,
  rightrule=38pt,
colback=green!10!white, 
colframe=green!50!black, 
title=\thetcbcounter,
detach title,
overlay unbroken and first ={
    \node[%rotate=90,
          %minimum width=1cm,
          anchor=south,
          font=\sffamily\bfseries\tiny,
          %yshift=-10pt,
          yshift=-5pt,
          xshift=-20pt,
          white]
    at (frame.east) {\thetcbcounter};
  }
}


\usepackage{xcolor}

%\usepackage{hyperref}
%\hypersetup{
%  pdfauthor={Lily Asquith},
%  urlcolor=blue,
%  colorlinks=true,
%  linkcolor=blue,
%  bookmarks=true
%}

%---------------------------------------------%
%              LILY'S COLOURS           %
%---------------------------------------------%
\definecolor{Wash}{RGB}{204,204,204}
%\definecolor{Pinky}{RGB}{254,200,254}%violet
\definecolor{Pinky}{RGB}{219,	240,	253}%violet
\definecolor{Bluey}{RGB}{0,190,255}%deep sky blue
\definecolor{DarkGrey}{RGB}{28,66,137}%dar grey
\definecolor{SussexWhite}{RGB}{253,255,254}%dar grey
\definecolor{LightGray}{RGB}{184,184,255}
\definecolor{YesGreen}{RGB}{0,128,0}
\definecolor{NoRed}{RGB}{250,0,0}



\definecolor{myred}{RGB}{255,153,153}
\definecolor{myorange}{RGB}{255,204,153}
\definecolor{myyellow}{RGB}{255,255,153}
\definecolor{mygreen}{RGB}{153,255,153}
\definecolor{mycyan}{RGB}{153,255,255}
\definecolor{myblue}{RGB}{153,204,255}
\definecolor{myviolet}{RGB}{153,153,255}
\definecolor{mypurple}{RGB}{204,153,255}
\definecolor{mypink}{RGB}{255,204,255}
\definecolor{mycoral}{RGB}{255,153,204}

%-----------------------------------------------------%
%              LILY'S COLUMN TYPES          %
%-----------------------------------------------------%
\newcolumntype{a}{>{\raggedright\arraybackslash}l}	
\newcolumntype{q}{>{\raggedright\arraybackslash}m{8cm}} 

%--------------------------------------------%
%              LILY'S SYMBOLS          %
%--------------------------------------------%
\newcommand{\dfinger}{\large{\textcolor{black}{\ding{43}}}\scriptsize}
\newcommand{\dstar}{\large{\textcolor{black}{\ding{76}}}\scriptsize}
\newcommand{\dwrite}{\large{\textcolor{black}{\ding{45}}}\scriptsize}
\newcommand{\ddiamond}{\small{\textcolor{DarkGrey}{\ding{117}}}\scriptsize}
\newcommand{\ddiamondwhite}{\small{\textcolor{SussexWhite}{\ding{117}}}\scriptsize}
\newcommand{\experiment}{\small{\textcolor{magenta}{\faCogs }}\scriptsize}
\newcommand{\watchit}{\textcolor{blue}{ \faYoutube}}


\makeatletter
\newcommand\notsotiny{\@setfontsize\notsotiny{6.5}{7.5}}
\makeatother


% 
\title[ Intro to Quantum Physics]{Intro to Quantum Physics F3241}
%\subtitle{\textbf{Part 4: The Photoelectric Effect}}
\author[Dr Lily Asquith (Lily)]{ Dr Lily Asquith (Lily)}
\date[Week 4]{ Week 4}
\logo{
\includegraphics[width=1.5cm]{../../utils/uslogo.jpg}
}


\begin{document}


\begin{frame}
\titlepage
\end{frame} 


\section{I2Q Part 4: The Photoelectric Effect}

 
 %-----------------------------------------------------------%
 % LECTURE 1
 %---------------------------------------------------------




\begin{frame}{Recap the Ultraviolet Catastrophe}
\small

\includegraphics[scale=0.45]{uvcatas-recap}
\end{frame}




\begin{frame}{The Photoelectric effect}
\small
\includegraphics[scale=0.45]{peform}
\end{frame}

\subsection{Hertz}

\begin{frame}{The work of Hertz, 1987}
\small
\includegraphics[scale=0.4]{hertz1}
\end{frame}

\begin{frame}{The work of Hertz}
\small
\includegraphics[scale=0.4]{hertz2}
\end{frame}

\begin{frame}{The work of Hertz}
\small
\includegraphics[scale=0.4]{hertz3}
\end{frame}

\begin{frame}{The work of Hertz}
\small
\includegraphics[scale=0.4]{hertz4}
\end{frame}

\begin{frame}{The work of Hertz}
\small
\includegraphics[scale=0.4]{hertz5}
\end{frame}

\begin{frame}{The work of Hertz}
\small
\includegraphics[scale=0.4]{hertz6}
\end{frame}
%-----------------------------------------------------
%     LECTURE 2
%-----------------------------------------------------

\subsection{The Photoelectric Effect}


\begin{frame}{Lenard's setup}
\small
\includegraphics[scale=0.4]{setup}
\end{frame}

\begin{frame}{What do we expect?}
\small
\includegraphics[scale=0.4]{expectations}
\end{frame}

\begin{frame}{What do we observe?}
\small
\includegraphics[scale=0.4]{observe1}
\end{frame}
 
 
 \begin{frame}{What do we observe?}
\small
\includegraphics[scale=0.4]{observe2}
\end{frame}

\begin{frame}{What do we observe?}
\small
\includegraphics[scale=0.4]{observe3}
\end{frame}


\begin{frame}{What else do we observe?}
\small
\includegraphics[scale=0.4]{observe4}
\end{frame}


\begin{frame}{The photoelectric effect}
\small
\begin{center}
\includegraphics[scale=0.4]{ehf}
\end{center}
\end{frame}


\begin{frame}{The formula}
\Huge
\begin{center}
$eV_{0} = hf = KE_{max} + \Phi$
\end{center}
\end{frame}

\begin{frame}{What does it mean?}
\small
\includegraphics[scale=0.4]{observe-meaning}
\end{frame}


\begin{frame}{How it all fits together so far}
\small
\includegraphics[scale=0.4]{sofar}
\end{frame}

\begin{frame}{Problems on The Electron, Radiation, Photoelectric Effect}
\small

\end{frame}

\begin{frame}{ASIDE: letting go of classical physics }
\small

\includegraphics[scale=0.45]{rj1}
\end{frame}


\begin{frame}{ASIDE: letting go of classical physics }
\small

\includegraphics[scale=0.45]{rj2}
\end{frame}
\begin{frame}{ASIDE: letting go of classical physics }
\small

\includegraphics[scale=0.45]{rj3}
\end{frame}
\begin{frame}{ASIDE: letting go of classical physics }
\small

\includegraphics[scale=0.45]{rj4}
\end{frame}

\end{document}