% preamble and style file for M&R lecture slides
\documentclass[11.5pt,sans,english]{beamer}

\usetheme{EastLansing}
\usecolortheme{lily}

\usepackage[most]{tcolorbox}

\usepackage{verbatim}
%\usepackage{ulem}
%\usepackage{fontawesome}
%\usepackage{tikz}
%\usepackage{pifont}
%\usepackage{tabularx}
\usepackage{array,booktabs,xcolor,colortbl,multirow,rotating,amssymb}
%\usepackage{amsmath}
% \usepackage{vwcol}
% \usepackage[T1]{fontenc}

  
\newcommand\vect[1]{\underline{\mathbf{#1}}}
\newcommand\unitvect[1]{\hat{\boldsymbol{#1}}}
%\newcommand\hatdot[1] { \hat{ \dot{ \boldsymbol{#1} } } }

\newtcbox
{\keyc}{on line,arc=2pt, colback=yellow!30!white, colframe=yellow!30!black, before upper={\rule[-3pt]{0pt}{10pt} },boxrule=1pt,boxsep=0pt,left=6pt,right=6pt,top=2pt,bottom=2pt,}

\newtcbox
{\keyb}{on line,arc=1pt, colback=blue!30!white, colframe=blue!30!black, before upper={\rule[-3pt]{0pt}{10pt} },boxrule=1pt,boxsep=0pt,left=6pt,right=6pt,top=2pt,bottom=2pt,}

\newtcbox
{\keyl}{on line,arc=1pt, colback=pink!30!white, colframe=blue!30!black, before upper={\rule[-3pt]{0pt}{10pt} },boxrule=1pt,boxsep=0pt,left=6pt,right=6pt,top=2pt,bottom=2pt,}

\newtcbox
{\keyw}{on line,arc=1pt, colback=red!30!white, colframe=blue!30!black, before upper={\rule[-3pt]{0pt}{10pt} },boxrule=1pt,boxsep=0pt,left=6pt,right=6pt,top=2pt,bottom=2pt,}

\newtcbox
{\keya}{on line,arc=1pt, colback=purple!30!white, colframe=blue!30!black, before upper={\rule[-3pt]{0pt}{10pt} },boxrule=1pt,boxsep=0pt,left=6pt,right=6pt,top=2pt,bottom=2pt,}

\newtcbox[auto counter,number within=section]
{keyf}
{
enhanced,
on line,
  boxsep=0pt,
  left=6pt,right=6pt,top=2pt,bottom=2pt,
  arc=5pt,
  boxrule=1pt,
  rightrule=38pt,
colback=green!10!white, 
colframe=green!50!black, 
title=\thetcbcounter,
detach title,
overlay unbroken and first ={
    \node[%rotate=90,
          %minimum width=1cm,
          anchor=south,
          font=\sffamily\bfseries\tiny,
          %yshift=-10pt,
          yshift=-5pt,
          xshift=-20pt,
          white]
    at (frame.east) {\thetcbcounter};
  }
}


\usepackage{xcolor}

%\usepackage{hyperref}
%\hypersetup{
%  pdfauthor={Lily Asquith},
%  urlcolor=blue,
%  colorlinks=true,
%  linkcolor=blue,
%  bookmarks=true
%}

%---------------------------------------------%
%              LILY'S COLOURS           %
%---------------------------------------------%
\definecolor{Wash}{RGB}{204,204,204}
%\definecolor{Pinky}{RGB}{254,200,254}%violet
\definecolor{Pinky}{RGB}{219,	240,	253}%violet
\definecolor{Bluey}{RGB}{0,190,255}%deep sky blue
\definecolor{DarkGrey}{RGB}{28,66,137}%dar grey
\definecolor{SussexWhite}{RGB}{253,255,254}%dar grey
\definecolor{LightGray}{RGB}{184,184,255}
\definecolor{YesGreen}{RGB}{0,128,0}
\definecolor{NoRed}{RGB}{250,0,0}



\definecolor{myred}{RGB}{255,153,153}
\definecolor{myorange}{RGB}{255,204,153}
\definecolor{myyellow}{RGB}{255,255,153}
\definecolor{mygreen}{RGB}{153,255,153}
\definecolor{mycyan}{RGB}{153,255,255}
\definecolor{myblue}{RGB}{153,204,255}
\definecolor{myviolet}{RGB}{153,153,255}
\definecolor{mypurple}{RGB}{204,153,255}
\definecolor{mypink}{RGB}{255,204,255}
\definecolor{mycoral}{RGB}{255,153,204}

%-----------------------------------------------------%
%              LILY'S COLUMN TYPES          %
%-----------------------------------------------------%
\newcolumntype{a}{>{\raggedright\arraybackslash}l}	
\newcolumntype{q}{>{\raggedright\arraybackslash}m{8cm}} 

%--------------------------------------------%
%              LILY'S SYMBOLS          %
%--------------------------------------------%
\newcommand{\dfinger}{\large{\textcolor{black}{\ding{43}}}\scriptsize}
\newcommand{\dstar}{\large{\textcolor{black}{\ding{76}}}\scriptsize}
\newcommand{\dwrite}{\large{\textcolor{black}{\ding{45}}}\scriptsize}
\newcommand{\ddiamond}{\small{\textcolor{DarkGrey}{\ding{117}}}\scriptsize}
\newcommand{\ddiamondwhite}{\small{\textcolor{SussexWhite}{\ding{117}}}\scriptsize}
\newcommand{\experiment}{\small{\textcolor{magenta}{\faCogs }}\scriptsize}
\newcommand{\watchit}{\textcolor{blue}{ \faYoutube}}


\makeatletter
\newcommand\notsotiny{\@setfontsize\notsotiny{6.5}{7.5}}
\makeatother


% 
\title[ Intro to Quantum Physics]{Intro to Quantum Physics F3241}
%\subtitle{\textbf{Part 4: The Photoelectric Effect}}
\author[Dr Lily Asquith (Lily)]{ Dr Lily Asquith (Lily)}
\date[Week 5]{ Week 5}
\logo{
\includegraphics[width=1.5cm]{../../utils/uslogo.jpg}
}


\begin{document}


\begin{frame}
\titlepage
\end{frame} 


\section{I2Q Part 5: Compton Scattering}

 
 \begin{frame}{Warm up!}
\small

$E_{\gamma} = hf_{\gamma} = \frac{hc_{\gamma} }{\lambda_{\gamma} }$\\

$E^2_{\gamma} = m_{\gamma} ^2 c_{\gamma} ^{4} +p_{\gamma} ^2c_{\gamma} ^2 $\\

$\frac{ h c_{\gamma} }{ \lambda_{\gamma} } = p_{\gamma} c_{\gamma} $\\

Momentum $p_{\gamma} = \frac{ h}{ \lambda_{\gamma} } $ \\
%\includegraphics[scale=0.45]{uvcatas-recap}

What about electrons?\\
\end{frame}


 %-----------------------------------------------------------%
 % LECTURE 1
 %---------------------------------------------------------

\subsection{Diffraction \& x-rays}


\begin{frame}{Diffraction \& x-rays}
\small
Shining light on a screen with a hole in it is really strong evidence that light is an EM wave.\\[1ex]

Remember Hertz showed this for radio waves, and then also sparked the photoelectric experiment, which makes radiation look like it is made of particles rather than waves...\\

You will do wave mechanics next term with David Seery, but let's get a little taster now, just enough that we can  for a sensible thought process about what is happening.\\

All waves interfere. If you go to a gig, try slowly moving horizontally between the stage and the sound rig. You will notice a change in both the quality and volume depending on where you are positioned. (Sound waves are not the same as EM waves: they are pressure waves).\\

For EM waves (aka radiation) they leave a source in every direction, spherically.\\

If we mess with that by placing a straight line in its path, we will cause the radiation to interfere with itself\\

At a screen place some distance away, what is happening?\\

When light shines on a surface, we can see it. It reflects off the surface and back to our eyes. What does that mean? Is it the same light?\\

Why does this jumper look blue?\\

Different things can happen when light is incident on a material. \\

The light is scattered from the material\\
- if it doesn't have enough energy to free an electron this will always happen. It will collide with an electron and conservation of energy means it must be scattered back in some direction.
- if it does have enough energy to free an electron this can also happen. It will pay the tax, free the electron, and then take the remaining energy off with it.\\

The light is absorbed by the material, having its entire energy redistributed as a tax to free the bound electron, and the freed electron energy.\\

The light is absorbed by the material, but instead of freeing a loosely bound electron, it makes a tightly bound one less loosely bound (excited). This guy will immediately drop back down into its unexcited ground state, and that energy will be released as light. 

The light passes through the material, as if it were not there. Or some of it does...



%\includegraphics[scale=0.45]{uvcatas-recap}
\end{frame}{Diffraction}


\begin{frame}
\small



\end{frame}




\end{document}