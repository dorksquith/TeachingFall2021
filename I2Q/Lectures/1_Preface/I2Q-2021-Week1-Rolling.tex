% preamble and style file for M&R lecture slides
\documentclass[11.5pt,sans,english]{beamer}

\usetheme{EastLansing}
\usecolortheme{lily}

\usepackage[most]{tcolorbox}

\usepackage{verbatim}
%\usepackage{ulem}
%\usepackage{fontawesome}
%\usepackage{tikz}
%\usepackage{pifont}
%\usepackage{tabularx}
\usepackage{array,booktabs,xcolor,colortbl,multirow,rotating,amssymb}
%\usepackage{amsmath}
% \usepackage{vwcol}
% \usepackage[T1]{fontenc}

  
\newcommand\vect[1]{\underline{\mathbf{#1}}}
\newcommand\unitvect[1]{\hat{\boldsymbol{#1}}}
%\newcommand\hatdot[1] { \hat{ \dot{ \boldsymbol{#1} } } }

\newtcbox
{\keyc}{on line,arc=2pt, colback=yellow!30!white, colframe=yellow!30!black, before upper={\rule[-3pt]{0pt}{10pt} },boxrule=1pt,boxsep=0pt,left=6pt,right=6pt,top=2pt,bottom=2pt,}

\newtcbox
{\keyb}{on line,arc=1pt, colback=blue!30!white, colframe=blue!30!black, before upper={\rule[-3pt]{0pt}{10pt} },boxrule=1pt,boxsep=0pt,left=6pt,right=6pt,top=2pt,bottom=2pt,}

\newtcbox
{\keyl}{on line,arc=1pt, colback=pink!30!white, colframe=blue!30!black, before upper={\rule[-3pt]{0pt}{10pt} },boxrule=1pt,boxsep=0pt,left=6pt,right=6pt,top=2pt,bottom=2pt,}

\newtcbox
{\keyw}{on line,arc=1pt, colback=red!30!white, colframe=blue!30!black, before upper={\rule[-3pt]{0pt}{10pt} },boxrule=1pt,boxsep=0pt,left=6pt,right=6pt,top=2pt,bottom=2pt,}

\newtcbox
{\keya}{on line,arc=1pt, colback=purple!30!white, colframe=blue!30!black, before upper={\rule[-3pt]{0pt}{10pt} },boxrule=1pt,boxsep=0pt,left=6pt,right=6pt,top=2pt,bottom=2pt,}

\newtcbox[auto counter,number within=section]
{keyf}
{
enhanced,
on line,
  boxsep=0pt,
  left=6pt,right=6pt,top=2pt,bottom=2pt,
  arc=5pt,
  boxrule=1pt,
  rightrule=38pt,
colback=green!10!white, 
colframe=green!50!black, 
title=\thetcbcounter,
detach title,
overlay unbroken and first ={
    \node[%rotate=90,
          %minimum width=1cm,
          anchor=south,
          font=\sffamily\bfseries\tiny,
          %yshift=-10pt,
          yshift=-5pt,
          xshift=-20pt,
          white]
    at (frame.east) {\thetcbcounter};
  }
}


\usepackage{xcolor}

%\usepackage{hyperref}
%\hypersetup{
%  pdfauthor={Lily Asquith},
%  urlcolor=blue,
%  colorlinks=true,
%  linkcolor=blue,
%  bookmarks=true
%}

%---------------------------------------------%
%              LILY'S COLOURS           %
%---------------------------------------------%
\definecolor{Wash}{RGB}{204,204,204}
%\definecolor{Pinky}{RGB}{254,200,254}%violet
\definecolor{Pinky}{RGB}{219,	240,	253}%violet
\definecolor{Bluey}{RGB}{0,190,255}%deep sky blue
\definecolor{DarkGrey}{RGB}{28,66,137}%dar grey
\definecolor{SussexWhite}{RGB}{253,255,254}%dar grey
\definecolor{LightGray}{RGB}{184,184,255}
\definecolor{YesGreen}{RGB}{0,128,0}
\definecolor{NoRed}{RGB}{250,0,0}



\definecolor{myred}{RGB}{255,153,153}
\definecolor{myorange}{RGB}{255,204,153}
\definecolor{myyellow}{RGB}{255,255,153}
\definecolor{mygreen}{RGB}{153,255,153}
\definecolor{mycyan}{RGB}{153,255,255}
\definecolor{myblue}{RGB}{153,204,255}
\definecolor{myviolet}{RGB}{153,153,255}
\definecolor{mypurple}{RGB}{204,153,255}
\definecolor{mypink}{RGB}{255,204,255}
\definecolor{mycoral}{RGB}{255,153,204}

%-----------------------------------------------------%
%              LILY'S COLUMN TYPES          %
%-----------------------------------------------------%
\newcolumntype{a}{>{\raggedright\arraybackslash}l}	
\newcolumntype{q}{>{\raggedright\arraybackslash}m{8cm}} 

%--------------------------------------------%
%              LILY'S SYMBOLS          %
%--------------------------------------------%
\newcommand{\dfinger}{\large{\textcolor{black}{\ding{43}}}\scriptsize}
\newcommand{\dstar}{\large{\textcolor{black}{\ding{76}}}\scriptsize}
\newcommand{\dwrite}{\large{\textcolor{black}{\ding{45}}}\scriptsize}
\newcommand{\ddiamond}{\small{\textcolor{DarkGrey}{\ding{117}}}\scriptsize}
\newcommand{\ddiamondwhite}{\small{\textcolor{SussexWhite}{\ding{117}}}\scriptsize}
\newcommand{\experiment}{\small{\textcolor{magenta}{\faCogs }}\scriptsize}
\newcommand{\watchit}{\textcolor{blue}{ \faYoutube}}


\makeatletter
\newcommand\notsotiny{\@setfontsize\notsotiny{6.5}{7.5}}
\makeatother


% 
\title[ Intro to Quantum Physics]{Intro to Quantum Physics F3241}
%\subtitle{\textbf{Part 1: Preface}}
\author[Dr Lily Asquith (Lily)]{ Dr Lily Asquith (Lily)}
\date[27 Sep - 01 Oct 2021]{ 27 Sep - 01 Oct 2021 (Week 1)}
\logo{
\includegraphics[width=1.5cm]{../../utils/uslogo.jpg}
}


\begin{document}


\begin{frame}
\titlepage
\end{frame} 

 %-----------------------------------------------------------%
 % 1 Kinematics                                                 %
 %-----------------------------------------------------------%
\section{I2Q Part 1: Preface}
\begin{frame}
\frametitle{Preface} 
\normalsize

This week's topics:\\[3ex]

\begin{itemize}
\item[1.1] Units, prefixes, and dimensionality\\[3ex]
\item[1.2] Constants\\[3ex]
\item[1.3] Tips \& Tricks\\[3ex]
\end{itemize}

Your homework questions for this week are on canvas - please complete these by the end of the week.
\end{frame} 
 

%-----------------------------------------------------
%     LECTURE 3
%-----------------------------------------------------

\subsection{Tips \& Tricks}

 \begin{frame}{Significant figures}

a) 4.0906060 : this has 8 significant figures (8SF)\\[1ex]
b) 0.0906060 : this has 6SF\\[1ex]
c) 400.0906060 : this has 10 SF\\[1ex]


\end{frame}


%\begin{frame}{Precision}
%1. Number of significant figures:
%a) 4.0906060 : this has 8 significant figures (8SF)
%b) 0.0906060 : this has 6SF
%c) 400.0906060 : this has 10 SF
%
%Q1) Why does b have 6SF, and not 8?
%A1) The first two figures, 0.0, are not significant because we do not need them to completely describe the number. Easier to see in scientific notation:
%0.0906060 => 9.06060 x 10^{-2} : 6SF
%No additional information is stored in the preceding zeros, therefore they are not significant.
%
%Q2) Why is the trailing zero significant?
%A2) Well, zero is a number. Think about the differences between these:
%4.0906060 : is the result of rounding up any values between 4.09060595 and 4.09060604
%4.090606 : is the result of rounding up any values between 4.0906055 and 4.0906064
%More information is stored in the number with a trailing zero, therefore it is significant.
%
%2. DURING a calculation, it is best not to round up at all. Obviously your calculator will, but most of the problems you will encounter are designed to expect this. Avoid using pen and paper. Using python (or matlab) is recommended - you will receive instruction on this.
%
%3. FOR THE FINAL ANSWER, give an appropriate number of significant figures:
%- Example 1: addition/subtraction
%138.7192 - 4.601 = 134.1182 (calculator)
%                             = 134.118 (final answer, underlined)
%
%So, we don't use more "Right SF"  in the answer than in any part of the question
%
%- Example 2: multiplication/division
%3.214*4.74 = 15.23436 (calculator)
%                  =  15.2 (final answer, underlined)
%
%So, we don't use more "Total SF" in the answer than in any part of the question
%
%\end{frame}






 
\end{document}