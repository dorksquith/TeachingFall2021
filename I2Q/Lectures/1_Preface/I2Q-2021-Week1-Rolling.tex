% preamble and style file for M&R lecture slides
\documentclass[11.5pt,sans,english]{beamer}

\usetheme{EastLansing}
\usecolortheme{lily}

\usepackage[most]{tcolorbox}

\usepackage{verbatim}
%\usepackage{ulem}
%\usepackage{fontawesome}
%\usepackage{tikz}
%\usepackage{pifont}
%\usepackage{tabularx}
\usepackage{array,booktabs,xcolor,colortbl,multirow,rotating,amssymb}
%\usepackage{amsmath}
% \usepackage{vwcol}
% \usepackage[T1]{fontenc}

  
\newcommand\vect[1]{\underline{\mathbf{#1}}}
\newcommand\unitvect[1]{\hat{\boldsymbol{#1}}}
%\newcommand\hatdot[1] { \hat{ \dot{ \boldsymbol{#1} } } }

\newtcbox
{\keyc}{on line,arc=2pt, colback=yellow!30!white, colframe=yellow!30!black, before upper={\rule[-3pt]{0pt}{10pt} },boxrule=1pt,boxsep=0pt,left=6pt,right=6pt,top=2pt,bottom=2pt,}

\newtcbox
{\keyb}{on line,arc=1pt, colback=blue!30!white, colframe=blue!30!black, before upper={\rule[-3pt]{0pt}{10pt} },boxrule=1pt,boxsep=0pt,left=6pt,right=6pt,top=2pt,bottom=2pt,}

\newtcbox
{\keyl}{on line,arc=1pt, colback=pink!30!white, colframe=blue!30!black, before upper={\rule[-3pt]{0pt}{10pt} },boxrule=1pt,boxsep=0pt,left=6pt,right=6pt,top=2pt,bottom=2pt,}

\newtcbox
{\keyw}{on line,arc=1pt, colback=red!30!white, colframe=blue!30!black, before upper={\rule[-3pt]{0pt}{10pt} },boxrule=1pt,boxsep=0pt,left=6pt,right=6pt,top=2pt,bottom=2pt,}

\newtcbox
{\keya}{on line,arc=1pt, colback=purple!30!white, colframe=blue!30!black, before upper={\rule[-3pt]{0pt}{10pt} },boxrule=1pt,boxsep=0pt,left=6pt,right=6pt,top=2pt,bottom=2pt,}

\newtcbox[auto counter,number within=section]
{keyf}
{
enhanced,
on line,
  boxsep=0pt,
  left=6pt,right=6pt,top=2pt,bottom=2pt,
  arc=5pt,
  boxrule=1pt,
  rightrule=38pt,
colback=green!10!white, 
colframe=green!50!black, 
title=\thetcbcounter,
detach title,
overlay unbroken and first ={
    \node[%rotate=90,
          %minimum width=1cm,
          anchor=south,
          font=\sffamily\bfseries\tiny,
          %yshift=-10pt,
          yshift=-5pt,
          xshift=-20pt,
          white]
    at (frame.east) {\thetcbcounter};
  }
}


\usepackage{xcolor}

%\usepackage{hyperref}
%\hypersetup{
%  pdfauthor={Lily Asquith},
%  urlcolor=blue,
%  colorlinks=true,
%  linkcolor=blue,
%  bookmarks=true
%}

%---------------------------------------------%
%              LILY'S COLOURS           %
%---------------------------------------------%
\definecolor{Wash}{RGB}{204,204,204}
%\definecolor{Pinky}{RGB}{254,200,254}%violet
\definecolor{Pinky}{RGB}{219,	240,	253}%violet
\definecolor{Bluey}{RGB}{0,190,255}%deep sky blue
\definecolor{DarkGrey}{RGB}{28,66,137}%dar grey
\definecolor{SussexWhite}{RGB}{253,255,254}%dar grey
\definecolor{LightGray}{RGB}{184,184,255}
\definecolor{YesGreen}{RGB}{0,128,0}
\definecolor{NoRed}{RGB}{250,0,0}



\definecolor{myred}{RGB}{255,153,153}
\definecolor{myorange}{RGB}{255,204,153}
\definecolor{myyellow}{RGB}{255,255,153}
\definecolor{mygreen}{RGB}{153,255,153}
\definecolor{mycyan}{RGB}{153,255,255}
\definecolor{myblue}{RGB}{153,204,255}
\definecolor{myviolet}{RGB}{153,153,255}
\definecolor{mypurple}{RGB}{204,153,255}
\definecolor{mypink}{RGB}{255,204,255}
\definecolor{mycoral}{RGB}{255,153,204}

%-----------------------------------------------------%
%              LILY'S COLUMN TYPES          %
%-----------------------------------------------------%
\newcolumntype{a}{>{\raggedright\arraybackslash}l}	
\newcolumntype{q}{>{\raggedright\arraybackslash}m{8cm}} 

%--------------------------------------------%
%              LILY'S SYMBOLS          %
%--------------------------------------------%
\newcommand{\dfinger}{\large{\textcolor{black}{\ding{43}}}\scriptsize}
\newcommand{\dstar}{\large{\textcolor{black}{\ding{76}}}\scriptsize}
\newcommand{\dwrite}{\large{\textcolor{black}{\ding{45}}}\scriptsize}
\newcommand{\ddiamond}{\small{\textcolor{DarkGrey}{\ding{117}}}\scriptsize}
\newcommand{\ddiamondwhite}{\small{\textcolor{SussexWhite}{\ding{117}}}\scriptsize}
\newcommand{\experiment}{\small{\textcolor{magenta}{\faCogs }}\scriptsize}
\newcommand{\watchit}{\textcolor{blue}{ \faYoutube}}


\makeatletter
\newcommand\notsotiny{\@setfontsize\notsotiny{6.5}{7.5}}
\makeatother


% 
\title[ Intro to Quantum Physics]{Intro to Quantum Physics F3241}
%\subtitle{\textbf{Part 1: Preface}}
\author[Dr Lily Asquith (Lily)]{ Dr Lily Asquith (Lily)}
\date[27 Sep - 01 Oct 2021]{ 27 Sep - 01 Oct 2021 (Week 1)}
\logo{
\includegraphics[width=1.5cm]{../../utils/uslogo.jpg}
}


\begin{document}


\begin{frame}
\titlepage
\end{frame} 

 %-----------------------------------------------------------%
 % 1 Kinematics                                                 %
 %-----------------------------------------------------------%
\section{I2Q Part 1: Preface}
\begin{frame}
\frametitle{Preface} 
\normalsize

This week's topics:\\[3ex]

\begin{itemize}
\item[1.1] Units, prefixes, and dimensionality\\[3ex]
\item[1.2] Constants\\[3ex]
\item[1.3] Tips \& Tricks\\[3ex]
\end{itemize}

Your homework questions for this week are on canvas - please complete these by the end of the week.
\end{frame} 
 
 %-----------------------------------------------------------%
 % LECTURE 1
 %-----------------------------------------------------------%
 
 \subsection{Units, prefixes, and dimensionality}

\begin{frame}{Topic 1.1 : Units, prefixes, and dimensionality}
\small
\textbf{Aim:}\\
To prepare your minds for the strange things to come, by convincing you that things are already strange.\\[3ex]
\textbf{Method:}\\
We will think about how we measure the world around us now, and what led us to use these measurement techniques.\\[1ex]

\end{frame}


%  
\begin{frame}{Where did all this start?}
\small
To do physics, we need clear, agreed-upon definitions and systems of measurements.\\[1ex]
But, human beings started to do science before there was any such clarity...\\[1ex]
\begin{center}
\includegraphics[scale=0.75]{pegasus.jpg}
\includegraphics[scale=0.25]{absolutes.png}
\end{center}
\end{frame}

\begin{frame}{Length}
Measurements of length or distance tend to have roots in human body parts (and later, farming).\\
\begin{center}
\includegraphics[scale=0.2]{length.png}
\includegraphics[scale=0.2]{humancow.png}
\end{center}
\end{frame}


\begin{frame}{Time}
Measurements of time tend to have their roots in the things we observe in the sky.\\

\begin{center}
\includegraphics[scale=0.3]{time.png}
\end{center}
\end{frame}


\begin{frame}{Weight}
Measurements of weight tend to have their roots in things we can pick up.\\
\begin{center}
\includegraphics[scale=0.3]{weight.png}
\end{center}
\end{frame}

\begin{frame}{Units}
Thankfully, we now have the Syst\`eme International, with 7 base units: \\[1ex]
%\begin{columns}
%\begin{column}{0.5\textwidth}
\begin{itemize}
\item Metres, m\\ % The metre is currently defined as the length of the path travelled by light in a vacuum in 1/299 792 458 of a second. 
\item Seconds, s\\
\item Kilograms, kg\\ % lump of metal in parisien basement - > now defined in terms of fundamental constants

\item Amperes, A\\ %The ampere is defined by taking the fixed numerical value of the elementary charge e to be 1.602 176 634 � 10?19 when expressed in the unit C, which is equal to A?s, where the second is defined in terms of ??Cs, the unperturbed ground state hyperfine transition frequency of the caesium-133 atom
\item Kelvin, K\\ % The kelvin is now defined by fixing the numerical value of the Boltzmann constant k to 1.380649�10?23 J?K?1. Hence, one kelvin is equal to a change in the thermodynamic temperature T that results in a change of thermal energy kT by 1.380649�10?23 J.[1] The relation between kelvin and Celsius scales is TK = t�C + 273.15. On the Kelvin scale, pure water freezes at 273.15 K, and it boils at 373.15 K in 1 atm.  Absolute zero does not exist. https://www.sciencealert.com/after-a-century-of-debate-cooling-to-absolute-zero-has-been-declared-mathematically-impossible

\item Candela, cd\\
\item Mole, mol\\

\end{itemize}
%\end{column}
%\begin{column}{0.5\textwidth}
%\begin{center}
%\includegraphics[scale=0.45]{oldSI.png}
%\includegraphics[scale=0.45]{newSI.png}
%\end{center}
%\end{column}
%\end{columns}

%\tiny
%Figures: By Emilio Pisanty - Own work, CC BY-SA 4.0
\end{frame}



\begin{frame}{Lingering Issues}
\small
For some folks, the metric system has been hard to adopt.\\
\begin{center}
\includegraphics[scale=0.45]{metric_map.png}
\end{center}
\end{frame}





\begin{frame}{Lingering Issues}
\begin{center}
\includegraphics[scale=0.5]{uk-pint.png}
\includegraphics[scale=0.5]{us-pint.png}
\end{center}
\small
\begin{itemize}
\item 12 inches in a foot, 3 feet in a yard, 1760 yards in a mile. By the way, a mile is 8 furlongs. A fathom is 2 yards and a league is 3 miles. 
\item British pint 568 ml : 20 imperial fluid ounces, \textcolor{red}{ American pint 473 ml : 16 US ounces}
\item British Ton 2,240 pounds (20 hundredweight each of 8 stone each of 14 pounds). \textcolor{red}{American Ton 2,000 pounds}. \textcolor{blue}{Metric tonne 2204.62 pounds}
\end{itemize}

\end{frame}



\begin{frame}{Lingering Issues}
\begin{center}
\includegraphics[scale=0.25]{measure-like-a-brit.jpg}
\end{center}
\end{frame}

%\begin{frame}{Poll everywhere checkpoint }
%
%Which is longer: a fathom, a furlong, or a mile?\\[1ex]
%
%\fbox{\begin{minipage}{\textwidth}
%Use your phone to go to: \textcolor{blue}{pollev.com/ilovephysics}
%
%\end{minipage}}
%\vspace{2cm}
%  \end{frame}


%\begin{frame}{Amounts of things}
%At some point, human beings took the step of \textbf{quantifying} things. \\[1ex]
%
%\end{frame}



\begin{frame}{Counting Systems}
\small
Like with units, the choice is based on `common sense' rather than anything fundamental:\\
\begin{itemize}
\item Sexagesimal [60]
%\item Vigesimal [20]
\item Duodecimal [12] 
\item Decimal [10]
\item Hexadecimal [16]
\item Binary [2]
\end{itemize}

\textit{Are there an infinity of possible counting systems?}
\end{frame}

\begin{frame}{Length: standardising the metre}
\small
First the pendulum was used, them a fraction of the estimated circumference of the Earth.\\
1791 (-1799): The French National Assembly accepts the proposal that the new definition for the metre be equal to `\textit{one ten-millionth of the length of a quadrant along the Earth's meridian through Paris}'  (the distance from the equator to the north pole.)
\begin{center}
\includegraphics[scale=0.3]{metre.png}
\end{center}
\end{frame}

\begin{frame}{Time: standardising the second}
\begin{columns}
\begin{column}{0.65\textwidth}
\small
Up to 1400 CE: The earliest mechanical clocks had hours, sometimes divided into quarters, more rarely into 12 parts.\\[1ex]
Clocks with minutes appeared late 16th century, followed by seconds in 17th-18th centuries\\[1ex]
1862: The British Association for the Advancement of Science stated that `\textit{All people of science are agreed to use the second of mean solar time as the unit of time.}'
\end{column}
\begin{column}{0.35\textwidth}
\begin{center}
\includegraphics[scale=0.35]{elephant_clock.png}
\end{center}
\end{column}
\end{columns}
\end{frame}

\begin{frame}{Weight: standardising the kilogram}
\small
1795: the mass of 1 litre of water (at melting point).\\
There is a prototype kilogram in the custody of the International Bureau of Weights and Measures in Sevres, France. 
\begin{center}
\includegraphics[scale=0.3]{kilo.png}
\end{center}
\end{frame}



 
 \begin{frame}{Er, what has this got do with quantum physics? }

The measures and counting systems we use are arbitrary, and have developed organically in a pretty chaotic way.\\[1ex]

There is nothing fundamentally meaningful about a 1 metre distance, or the number 10.\\[1ex]

We can and will abandon standard measures and units in order to make calculations easier.\\[1ex]

There are things in this universe that do seem to be fundamentally meaningful.\\[1ex]

  \end{frame}


\begin{frame}{Prefixes}
\scriptsize
%\begin{center}
\begin{tabular}{c c c l}
\textbf{Prefix} &  \textbf{Symbol} & \textbf{Factor} & \textbf{Etymology}\\[1ex]
peta & P & $10^{15}$ & Greek: `five' \\[1ex] % DATA!
tera & T & $10^{12}$ & Greek: `monster' \\ [1ex]% LHC energies
giga & G & $10^9$ & Greek: `giant' \\ [1ex]% Particle masses
mega & M & $10^6$ & Greek: `big' \\[1ex]
kilo & k & $10^3$ & Greek: `thousand' \\[2ex]

zero &  & $10^{0}$ &  Arabic: `sifr : empty'  \\[2ex]

milli & m & $10^{-3}$ &  	Latin: `thousandth' \\[1ex]
micro & $\mu$ & $10^{-6}$ &  Greek: `small' \\[1ex]
nano & n & $10^{-9}$ &  	Greek: `dwarf' \\ [1ex]% wavelengths
pico & p & $10^{-12}$ &  	Spanish: `tiny bit'\\[1ex]
femto & f & $10^{-15}$ &  	Dano-Norwegian: `fifteen'\\ [1ex]% interaction cross sections

\end{tabular}
%\end{center}

\end{frame}



 \begin{frame}{Dimensionality : LMT}
To get a grip and do useful science, we should think about what it is important.\\
\begin{itemize}
\item Length [L]\\
\item Mass [M]\\
\item Time [T]\
\end{itemize}

The dimension of area is $L^2$. \\[1ex]
\begin{itemize}
\item[a] What is the dimension of volume?\\[1ex]
\item[b] What is the dimension of density (mass/volume)?\\[1ex]
\item[c] What is the dimension of acceleration?\\[1ex]
\item[d] What is the dimension of force?\\[1ex]
\item[e] What is the dimension of energy?\\[1ex]
\end{itemize}
\end{frame}
 

 \begin{frame}{End of part 1.1}
End of part 1.1 : Units, prefixes, and dimensionality\\[1ex]
Homework is on canvas.
\end{frame}


%-----------------------------------------------------
%     LECTURE 2
%-----------------------------------------------------

\subsection{Fundamental Constants \& Natural Units}
\begin{frame}{What is a fundamental constant?}

Things that change depending on where and when you measure them:\\

Things that do not change depending on where and when you measure them:\\


\end{frame}

\begin{frame}{The speed of light}

299792458 metres per second (exact). How can it be exact?\\
Def of metre, sec.

\end{frame}

\begin{frame}{The charge on an electron}
%1.602176634�10?19 C (exact) \\
%Def of Coulomb (Amp, sec) the coulomb is exactly $1/(1.602176634\times10^{?19})$ elementary charges.\\

\end{frame}

\begin{frame}{Planck's constant}
https://home.cern/news/news/engineering/lock-planck-kilogram-has-new-definition

\end{frame}

\begin{frame}{The Kelvin}

\end{frame}


\begin{frame}{Avogadro's number}

\end{frame}

\begin{frame}{The Systeme International}

\end{frame}

\begin{frame}{Natural Units}

\end{frame}


%-----------------------------------------------------
%     LECTURE 3
%-----------------------------------------------------

\subsection{Tips \& Tricks}

 \begin{frame}{Precision}
1. Number of significant figures:
%a) 4.0906060 : this has 8 significant figures (8SF)
%b) 0.0906060 : this has 6SF
%c) 400.0906060 : this has 10 SF
%
%Q1) Why does b have 6SF, and not 8?
%A1) The first two figures, 0.0, are not significant because we do not need them to completely describe the number. Easier to see in scientific notation:
%0.0906060 => 9.06060 x 10^{-2} : 6SF
%No additional information is stored in the preceding zeros, therefore they are not significant.
%
%Q2) Why is the trailing zero significant?
%A2) Well, zero is a number. Think about the differences between these:
%4.0906060 : is the result of rounding up any values between 4.09060595 and 4.09060604
%4.090606 : is the result of rounding up any values between 4.0906055 and 4.0906064
%More information is stored in the number with a trailing zero, therefore it is significant.
%
%2. DURING a calculation, it is best not to round up at all. Obviously your calculator will, but most of the problems you will encounter are designed to expect this. Avoid using pen and paper. Using python (or matlab) is recommended - you will receive instruction on this.
%
%3. FOR THE FINAL ANSWER, give an appropriate number of significant figures:
%- Example 1: addition/subtraction
%138.7192 - 4.601 = 134.1182 (calculator)
%                             = 134.118 (final answer, underlined)
%
%So, we don't use more "Right SF"  in the answer than in any part of the question
%
%- Example 2: multiplication/division
%3.214*4.74 = 15.23436 (calculator)
%                  =  15.2 (final answer, underlined)
%
%So, we don't use more "Total SF" in the answer than in any part of the question

\end{frame}
\begin{frame}{Notation}

\end{frame}

\begin{frame}{Using python as a calculator}

\end{frame}

\begin{frame}{Equating things}

\end{frame}

\begin{frame}{Dimensional Analysis}

\end{frame}

\begin{frame}{Asking Questions}

\end{frame}

\begin{frame}{The Wheel house (or memory palace)}

\end{frame}


\begin{frame}{Poll Everywhere Checkpoint (\textcolor{blue}{pollev.com/ilovephysics})}
\notsotiny
%\includegraphics[scale=0.4]{GraphQuiz}

\end{frame}






 
\end{document}