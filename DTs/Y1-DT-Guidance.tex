\documentclass[11.5pt,sans,english]{article}
\usepackage[utf8]{inputenc}
\renewcommand{\familydefault}{\sfdefault}
 \usepackage[a4paper, total={7in, 9in}]{geometry}
\usepackage{listings}
\usepackage{color,colortbl}

\usepackage{fancyvrb}
\usepackage{xcolor}
\usepackage{tikz}
\usetikzlibrary{shapes,arrows}
\usepackage{multirow}
\usepackage{hyperref}
\usepackage{caption}
\usepackage{array}
\usepackage{tabularx}
\usepackage{wrapfig}

\usepackage{graphicx}
\graphicspath{{figs/}}

\hypersetup{
  pdfauthor={Lily Asquith},
  urlcolor=blue,
  colorlinks=true,
  linkcolor=purple,
  bookmarks=true
}

\definecolor{dkgreen}{rgb}{0,0.6,0}
\definecolor{gray}{rgb}{0.5,0.5,0.5}
\definecolor{mauve}{rgb}{0.58,0,0.82}

\lstset{frame=tb,
  language=python,
  aboveskip=3mm,
  belowskip=3mm,
  showstringspaces=false,
  columns=flexible,
  basicstyle={\small\ttfamily},
  numbers=none,
  numberstyle=\tiny\color{gray},
  keywordstyle=\color{blue},
  commentstyle=\color{dkgreen},
  stringstyle=\color{mauve},
  breaklines=true,
  breakatwhitespace=true,
  tabsize=3
}


\title{Y1 DT Guidance}
\author{Lily Asquith}
\date{Sept 2021}

\begin{document}

\tikzstyle{a} = [rectangle, draw, fill=green!20,  node distance=2.5cm,text width=5em, text centered, rounded corners, minimum height=2em]
\tikzstyle{line} = [draw, -latex']

\tikzstyle{restrictedcaf} = [draw, ellipse,fill=red!20, node distance=2.5cm,minimum height=2em]
    
    
    

%\maketitle
%
%\begin{abstract}
%This document details the containment studies for the NOvA FD.
%\end{abstract}
%
%
%\tableofcontents

%\newpage

\section{Timetable}

\includegraphics[scale=0.6]{DT-timetable}

\section{Face-to-face workshops}

\subsection{Arrival and setup}
Arrive at the allocated teaching space on the hour.\\
Login to the computer and have the  workshop questions for both MM1 and M\&R ready to share on the projector.\\
Login to poll everywhere ilovephysics and activate the poll "Workshop N questions". Share this on the projector.
Normally teaching sessions run from :00 to :50, but considering the current health and safety stuff for the pandemic, I would aim to start at 5 minutes past the hour and finish at 10 minutes to the hour.\\

\subsection{Introduction}
Explain that the workshop should be roughly 20 mins each on MM1 and M\&R\\
Ask the students to indicate any questions they are struggling with on the poll. They will almost certainly pick those that they are asked to submit the answers for for grading.\\
Tell the students that they should discuss and work together if they want to, and it is also fine to work alone if they wish.\\
Tell them that if they would like any answers checked or want anything clarified, to put their hand up\\
Walk around the room to see if anyone needs a paper copy of the questions or has any questions\\

\subsection{Helping students}
If they show you an answer and it is numerically correct, let them know that it is. This is also fine to do for the highlighted questions for submission.\\
If they show you an answer and it is incorrect, read the question aloud with them and then go through their working step by step.\\
Remember that there is usually more than one way of solving a problem. Their method may not be wrong, but to save you having to determine this on the fly, a good technique is to say `right, this may well be a good method for solving this problem, but may I share with you the method I prefer?' - this way you can stick to the worked solutions we have provided you with.
If a student say they really don't understand, or really are struggling to make a start, check with them if they have attempted the adaptive practice assignments on canvas. Those assignments link the students to the sections of the eTextbook they should read. They may have had trouble getting on to canvas or setting up wiley - often I meet students who have almost no technical skills - if this is the case please get their email and email them and me so we can set up a time to get them going online. \\

\subsection{Finishing up}
Explain to the students that they should take a photo of their attempt at the highlighted questions and upload to the correct canvas page by the end of this week.\\

\section{Zoom workshops}


\subsection{Arrival and setup}
Join or start the zoom session on the hour.\\
There will be two breakout rooms, one for each topic.\\
Have the workshop questions for both MM1 and M\&R ready to share on zoom.\\
Login to poll everywhere ilovephysics and activate the poll "Workshop N questions". Share this on zoom.
Normally teaching sessions run from :00 to :50, but considering the current health and safety stuff for the pandemic, I would aim to start at 5 minutes past the hour and finish at 10 minutes to the hour.\\

\subsection{Introduction}
Explain that the workshop should be roughly 20 mins each on MM1 and M\&R\\
Ask the students to indicate any questions they are struggling with on the poll. They will almost certainly pick those that they are asked to submit the answers for for grading.\\
Tell the students that they should discuss and work together if they want to, and it is also fine to work alone if they wish.\\
Tell them that if they would like any answers checked or want anything clarified, to go to the relevant breakout room and to put their hand up\\
Join your respective breakout rooms\\

\subsection{Helping students}
If a student puts their hand up, say hi and ask them how you can help.\\
If they show/tell you an answer and it is numerically correct, let them know that it is. This is also fine to do for the highlighted questions for submission.\\
If they show/tell you an answer and it is incorrect, read the question aloud with them and then go through their working step by step.\\
Remember that there is usually more than one way of solving a problem. Their method may not be wrong, but to save you having to determine this on the fly, a good technique is to say `right, this may well be a good method for solving this problem, but may I share with you the method I prefer?' - this way you can stick to the worked solutions we have provided you with.\\
If a student say they really don't understand, or really are struggling to make a start, check with them if they have attempted the adaptive practice assignments on canvas. Those assignments link the students to the sections of the eTextbook they should read. They may have had trouble getting on to canvas or setting up wiley - often I meet students who have almost no technical skills - if this is the case please get their email and email them and me so we can set up a time to get them going online. \\

\subsection{Finishing up}
Explain to the students that they should take a photo of their attempt at the highlighted questions and upload to the correct canvas page by the end of this week.\\


\end{document}
